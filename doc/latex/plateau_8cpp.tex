\hypertarget{plateau_8cpp}{}\section{/\+Users/thibault/\+Google Drive/\+P\+R\+O\+G/\+S4/\+Projet/cpp/plateau.cpp File Reference}
\label{plateau_8cpp}\index{/\+Users/thibault/\+Google Drive/\+P\+R\+O\+G/\+S4/\+Projet/cpp/plateau.\+cpp@{/\+Users/thibault/\+Google Drive/\+P\+R\+O\+G/\+S4/\+Projet/cpp/plateau.\+cpp}}


Fichier source contenant les fonctions de la classe plateau.  


{\ttfamily \#include $<$iostream$>$}\newline
{\ttfamily \#include $<$string$>$}\newline
{\ttfamily \#include $<$vector$>$}\newline
{\ttfamily \#include \char`\"{}../h/plateau.\+h\char`\"{}}\newline


\subsection{Detailed Description}
Fichier source contenant les fonctions de la classe plateau. 

Pour le plateau, j\textquotesingle{}ai choisi de représenter la grille hexagonale comme un tableau de dimensions (2$\ast$taillecote)-\/1 on doit faire un affichage en hexagone, pour cela j\textquotesingle{}ai choisi le symbole en \char`\"{}ascii art\char`\"{} de l\textquotesingle{}hexagone suivant \+: \begin{DoxyVerb}  ____
 /    \
/      \
\      / 
 \____/
\end{DoxyVerb}
 Ce qui nous donne des grilles ressemblant à \+: \begin{DoxyVerb}        ____ 
       /    \
  ____/      \____
 /    \      /    \
/      \____/      \
\      /    \      /
 \____/------\____/
 /    \------/    \
/======\____/======\
\======/    \======/
 \____/------\____/
      \------/
       \____/
\end{DoxyVerb}
 Les pièces appartenant à des joueurs différents seront différenciées par le symbole qui les remplit. 